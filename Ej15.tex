\item Hallar una base de $V_1+V_2+V_3$ para los siguientes subespacios de $\R^5$.
    \begin{enumerate}
        \item $V_1=\overline{(1,1,2,0,1);(2,0,3,0,1)}, V_2=\overline{(-1,1,-2,1,1)}$ y $V_3=\overline{(0,1,0,1,1)}$.
            \begin{mdframed}[style=s]
                Sea $v\in V_1+V_2+V_3,v=v_1+v_2+v_3\quad v_1\in V_1,v_2\in V_2,v_3\in V_3$. Además, tenemos que
                \begin{center}
                    $v_1=a(1,1,2,0,1)+b(2,0,3,0,1)\quad a,b\in\R$\\
                    $v_2=c(-1,1,-2,1,1)\quad c\in\R$\\
                    $v_3=d(0,1,0,1,1)\quad d\in\R$
                \end{center}
                Por lo tanto,
                \begin{center}
                    $v=a(1,1,2,0,1)+b(2,0,3,0,1)+c(-1,1,-2,1,1)+d(0,1,0,1,1)\quad a,b,c,d\in\R$
                \end{center}
                Tenemos que $V_1+V_2+V_3=\overline{\{(1,1,2,0,1);(2,0,3,0,1);(-1,1,-2,1,1);(0,1,0,1,1)\}}$. Se puede comprobar que los vectores son li, entonces una posible base es
                \begin{center}
                    $B=\{(1,1,2,0,1);(2,0,3,0,1);(-1,1,-2,1,1);(0,1,0,1,1)\}$
                \end{center}
            \end{mdframed}
        \item $V_1=\overline{(1,1,2,0,1);(2,0,3,0,1)}, V_2=\overline{(1,0,-2,1,1)}$ y $V_3=\overline{(1,1,1,2,2)}$.
            \begin{mdframed}[style=s]
                Es un caso análogo al anterior, y se puede llegar a que una posible base entonces
                \begin{center}
                    $B=\{(1,1,2,0,1);(2,0,3,0,1);(1,0,-2,1,1);(1,1,1,2,2)\}$
                \end{center}
            \end{mdframed}
    \end{enumerate}