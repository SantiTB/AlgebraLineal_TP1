\item Sean $V_1=\{A\in\C^{2\times2}:a_{11}+a_{12}=0\}$ y $V_2=\{A\in\C^{2\times2}:a_{11}+a_{21}=0\}$
    \begin{enumerate}
        \item Probar que $V_1$ y $V_2$ son subespacios de $\C^{2\times2}$ (como $\C$-EV).
            \begin{mdframed}[style=s]
                \begin{itemize}
                    \item $V_1$:
                    \begin{enumerate}
                        \item $\begin{pmatrix}
                                0&0\\0&0
                            \end{pmatrix}\in V_1$ ya que $0+0=0$
                            \item Sean $A,B\in V_1$ 
                                \begin{center}
                                    $\to A+B=\begin{pmatrix}
                                        a_{11}&a_{12}\\a_{21}&a_{22}
                                    \end{pmatrix}+\begin{pmatrix}
                                        b_{11}&b_{12}\\b_{21}&b_{22}
                                    \end{pmatrix}=\begin{pmatrix}
                                        a_{11}+b_{11}&a_{12}+b_{12}\\a_{21}+b_{21}&a_{22}+b_{22}
                                    \end{pmatrix}$    
                                \end{center}
                                Y como $a_{11}+b_{11}+a_{12}+b_{12}=a_{11}+a_{12}+b_{11}+b_{12}=0+0=0\to A+B\in V_1$
                            \item Sea $\alpha\in\C, A\in V_1$
                                \begin{center}
                                    $\to\alpha A=\alpha\begin{pmatrix}
                                        a_{11}&a_{12}\\a_{21}&a_{22}
                                    \end{pmatrix}=\begin{pmatrix}
                                        \alpha a_{11}&\alpha a_{12}\\\alpha a_{21}&\alpha a_{22}
                                    \end{pmatrix}$    
                                \end{center}
                                de donde $\alpha a_{11}+\alpha a_{12}=\alpha(a_{11}+a_{12})=\alpha 0=0\to \alpha A\in V_1$
                        \end{enumerate}
                        Entonces, $V_1$ es subespacio de $\C^{2\times2}$
                    \item $V_2:$\\
                        La demostración es análoga a la anterior cambiando las entradas 12 por la 21.
                \end{itemize}
            \end{mdframed}
        \item Hallar $V_1\cap V_2$ y $V_1+V_2$.
            \begin{mdframed}[style=s]
                \begin{itemize}
                    \item $V_1\cap V_2=\{A\in\C^{2\times2}:a_{11}+a_{12}=0,a_{11}+a_{21}=0\}=\left\{A\in\C^{2\times2}:A=\begin{pmatrix}
                            a_{11}&-a_{11}\\-a_{11}&a_{22}    
                        \end{pmatrix}\right\}$
                        \begin{center}
                            $\to A=\overline{\left\{\begin{pmatrix}
                                1&-1\\-1&0
                            \end{pmatrix};\begin{pmatrix}
                                0&0\\0&1
                            \end{pmatrix}\right\}}$
                        \end{center}
                    \item $V_1+V_2=\{A\in\C^{2\times2}:A=X+Y,\quad X\in V_1,Y\in V_2\}$\\
                        Entonces, tengo que 
                        \begin{center}
                            $A=\begin{pmatrix}
                                x_{11}&-x_{11}\\x_{21}&x_{22}
                            \end{pmatrix}+\begin{pmatrix}
                                y_{11}&y_{12}\\-y_{11}&y_{22}
                            \end{pmatrix}=\begin{pmatrix}
                                x_{11}+y_{11}&-x_{11}+y_{12}\\x_{21}-y_{11}&x_{22}+y_{22}
                            \end{pmatrix}$    
                        \end{center}
                        Por lo tanto, $A$ es una combinación lineal de
                        \begin{center}
                            $A=x_{11}\begin{pmatrix}
                                1&-1\\0&0
                            \end{pmatrix}+x_{21}\begin{pmatrix}
                                0&0\\1&0
                            \end{pmatrix}+x_{22}\begin{pmatrix}
                                0&0\\0&1
                            \end{pmatrix}+y_{11}\begin{pmatrix}
                                1&0\\-1&0
                            \end{pmatrix}+y_{12}\begin{pmatrix}
                                0&1\\0&0
                            \end{pmatrix}+y_{22}\begin{pmatrix}
                                0&0\\0&1
                            \end{pmatrix}$\\
                            $A=x_{11}\begin{pmatrix}
                                1&-1\\0&0
                            \end{pmatrix}+x_{21}\begin{pmatrix}
                                0&0\\1&0
                            \end{pmatrix}+(x_{22}+y_{22})\begin{pmatrix}
                                0&0\\0&1
                            \end{pmatrix}+y_{11}\begin{pmatrix}
                                1&0\\-1&0
                            \end{pmatrix}+y_{12}\begin{pmatrix}
                                0&1\\0&0
                            \end{pmatrix}$
                        \end{center}
                        Se ve que la matriz que tiene como coeficiente a $x_{11}$ es una combinación lineal de las matrices cuyos coeficientes son $x_{21},y_{11}$ y $y_{12}$
                        \begin{center}
                            $\begin{pmatrix}
                                1&-1\\0&0
                            \end{pmatrix}=\begin{pmatrix}
                                0&0\\1&0
                            \end{pmatrix}+\begin{pmatrix}
                                1&0\\-1&0
                            \end{pmatrix}-\begin{pmatrix}
                                0&1\\0&0
                            \end{pmatrix}$
                        \end{center}
                        Entonces, se tiene que
                        \begin{center}
                            $A=x_{21}\begin{pmatrix}
                                0&0\\1&0
                            \end{pmatrix}+(x_{22}+y_{22})\begin{pmatrix}
                                0&0\\0&1
                            \end{pmatrix}+y_{11}\begin{pmatrix}
                                1&0\\-1&0
                            \end{pmatrix}+y_{12}\begin{pmatrix}
                                0&1\\0&0
                            \end{pmatrix}$
                        \end{center}
                    Estas 4 matrices son li, con lo que se concluye que una base de $V_1+V_2$ es 
                    \begin{center}
                        $B=\left\{\begin{pmatrix}
                            0&0\\1&0
                        \end{pmatrix};\begin{pmatrix}
                            0&0\\0&1
                        \end{pmatrix};\begin{pmatrix}
                            1&0\\-1&0
                        \end{pmatrix};\begin{pmatrix}
                            0&1\\0&0
                        \end{pmatrix}\right\}$    
                    \end{center}
                    Al tener 4 elementos, el subespacio es de dimensión 4. Entonces, 
                    \begin{center}
                        $V_1+V_2=\C^{2\times2}$    
                    \end{center}
                \end{itemize}
            \end{mdframed}
        \item Hallar las dimensiones de $V_1,V_2, V_1\cap V_2$ y $V_1+V_2$
            \begin{mdframed}[style=s]
                \begin{itemize}
                    \item Sea $A\in V_1$
                        \begin{center}
                            $\to A=\begin{pmatrix}
                                a_{11}&-a_{11}\\a_{21}&a_{22}
                            \end{pmatrix}=a_{11}\begin{pmatrix}
                                1&-1\\0&0
                            \end{pmatrix}+a_{21}\begin{pmatrix}
                                0&0\\1&0
                            \end{pmatrix}+a_{22}\begin{pmatrix}
                                0&0\\0&1
                            \end{pmatrix}$        
                        \end{center}
                        $A$ es combinación lineal de 3 matrices li, entonces $dim(V_1)=3$
                    \item Sea $A\in V_2$
                        \begin{center}
                            $\to A=\begin{pmatrix}
                                a_{11}&a_{12}\\-a_{11}&a_{22}
                            \end{pmatrix}=a_{11}\begin{pmatrix}
                                1&0\\-1&0
                            \end{pmatrix}+a_{21}\begin{pmatrix}
                                0&1\\0&0
                            \end{pmatrix}+a_{22}\begin{pmatrix}
                                0&0\\0&1
                            \end{pmatrix}$        
                        \end{center}
                        $A$ es combinación lineal de 3 matrices li, entonces $dim(V_2)=3$
                    \item Del inciso anterior se ve que $V_1\cap V_2$ está generado por 2 matrices li. Entonces $dim(V_1\cap V_2)=2$
                    \item $V_1+V_2=\C^{2\times2}\to dim(V_1+V_2)=4$
                \end{itemize}
            \end{mdframed}
    \end{enumerate}